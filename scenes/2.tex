\kfframes{2}{}                                    %% : 18 - : 22

\kfframes{13}{
\newcommand\ta{4} % ONE CUBED ...
\newcommand\tb{11} % OVER THE COMPLEX ...
\newcommand\tc{13} % fadeout
\newcommand\tz{14} % end

\begin{scope}[shift={(\halfwidth, \halfheight)}]

% draw the real line and the origin

% labelled origin - in
\kftween[in=0,duration=\dt]{0}{0.12}
\pgfmathsetmacro\r\pgfmathresult
\kftween[in=\tb,duration=\dt]{0.12}{0}
\pgfmathsetmacro\r{\t < \tb ? \r : \pgfmathresult}

\fill (0, 0) circle (\r);
\kftextin[in=0, duration=\dt, out=\tb]{0.6}{-pi/4}{$0$};
\kftextout[in=\tb, out=\tb+\dt]{0.6}{-pi/4}{$0$};


% 1 - this should look as if it was drawn by a \zin, but we're not in
% a \zgrid yet, so fake it
\kftween[in=\ta,duration=(\dt/2)]{0}{2}
\pgfmathsetmacro\l\pgfmathresult
\kftween[in=(\ta+\dt/2),duration=(\dt/2)]{0}{0.12}
\pgfmathsetmacro\r\pgfmathresult
\draw (0, 0) -- (\l, 0);
\fill (2, 0) circle (\r);
\kftextin[in=\ta+\dt, duration=\dt, offset fst=0.6, offset snd=-pi/4]{2}{0}{$1$};

%% % opacity tween to fake the grey 0.5
\kftween[in=\tc, duration=\dt]{1}{0.5}
\pgfmathsetmacro\o\pgfmathresult

\kftextin   [polar=false, in=\dt, duration=\dt, out=\tb+\dt]{\halfwidth-0.5}{0.5}{$\mathbb{R}$}
\kftexttween[polar=false, in=(\tb+\dt), duration=\dt, out=\tz]
 {\halfwidth-0.5}{+0.5}{$\mathbb{R}$}
 {\halfwidth-0.6}{+0.5}{$\operatorname{Re}$}
\kftextin   [polar=false, in=\tb+\dt, duration=\dt, out=\tz]{+0.6}{\halfheight-0.4}{$\operatorname{Im}$}


\kftween[in=0,duration=\dt]{-\halfwidth}{\halfwidth}
\pgfmathsetmacro\l\pgfmathresult
\draw[opacity=\o] (-\halfwidth, 0) -- (\l, 0);

\kftween[in=\tb,duration=\dt]{0}{\halfheight}
\pgfmathsetmacro\l\pgfmathresult
\draw[opacity=\o] (0, -\l) -- (0, \l);

%% % x^3 - 1 = 0 -> z^3 - 1 = 0
\kftextin [polar=false, in=2.0, duration=\dt, out=\tc]{0}{+2}{\colorbox{black}{\Large $\phantom{x^3 - 1 = 0}$}}
\kftextout[polar=false, in=\tc, out=\tc+2*\dt]{0}{+2}{\colorbox{black}{\Large $\phantom{x^3 - 1 = 0}$}}

\kftextin [polar=false, in=2.0, duration=\dt, out=\tc]{0}{+2}{\Large $\phantom{x}^3 - 1 = 0$}
\kftextout[polar=false, in=\tc, out=\tc+\dt]{0}{+2}{\Large $\phantom{x}^3 - 1 = 0$}


\kftextin[polar=false, in=2.0, duration=\dt, out=\tb]{0}{+2}{\Large $x\phantom{^3 - 1 = 0}$}
%  \kftextout[polar=false, in=\tb, duration=\dt]{\halfwidth}{\halfheight+2}{\Large $x\phantom{^3 - 1 = 0}$}
%  \kftextin[polar=false, in=\tb, duration=\dt, out=\tc]{\halfwidth}{\halfheight+2}{\Large $z\phantom{^3 - 1 = 0}$}
\kftexttween[polar=false, in=\tb, duration=\dt, out=\tc]
{0}{+2}{\Large $x\phantom{^3 - 1 = 0}$}
{0}{+2}{\Large $z\phantom{^3 - 1 = 0}$}
\end{scope}
}
             %% : 22 - : 35

\kfframes{10}{
  \newcommand\md{1.5}
  \newcommand\ar{0.98}
  \newcommand\delay{2.5}
  \newcommand\ta{9}
  \zgrid[unit=1.2]{

    \z{1}{0}
    \kftext[offset fst=0.6, offset snd=-pi/4]{\zunit}{0}{$1$};

    \ztween[out=\ta]{1}{0}{\md}{\ar}
    \ztween[in=\delay, out=\ta, arg origin=\ar]{\md}{\ar}{\md^2}{2*\ar}
    \ztween[in=2*\delay, out=\ta, arg origin=2*\ar]{\md^2}{2*\ar}{\md^3}{3*\ar}

    \kftextin[in=2*\dt, duration=\dt, out=\ta, offset fst=0.5, offset snd=0]{\zunit*\md}{\ar}{$z$};
    \kftextin[in=\delay+2*\dt, duration=\dt, out=\ta, offset fst=0.7, offset snd=pi/4]{\zunit*\md^2}{2*\ar}{$z^2$};
    \kftextin[in=2*\delay+2*\dt, duration=\dt, out=\ta, offset fst=0.6, offset snd=pi/2]{\zunit*\md^3}{3*\ar}{$z^3$};

    % outros
    \zout[in=\ta, out=\ta+\dt]{\md}{\ar}
    \zout[in=\ta, out=\ta+\dt, arg origin=\ar]{\md^2}{2*\ar}
    \zout[in=\ta, out=\ta+\dt, arg origin=2*\ar]{\md^3}{3*\ar}

    \kftextout[in=\ta, out=\ta+\dt, duration=\dt/2, offset fst=0.5, offset snd=0]{\zunit*\md}{\ar}{$z$};
    \kftextout[in=\ta, out=\ta+\dt, duration=\dt/2, offset fst=0.7, offset snd=pi/4]{\zunit*\md^2}{2*\ar}{$z^2$};
    \kftextout[in=\ta, out=\ta+\dt, duration=\dt/2, offset fst=0.6, offset snd=pi/2]{\zunit*\md^3}{3*\ar}{$z^3$};    
  }
}
    %% : 35 - : 45

\kfframes{1}{                                     %% : 45 - : 46
\kftween[in=0, duration=\dt]{1.2}{3}
\pgfmathsetmacro\u\pgfmathresult
\zgrid[unit=\u]{
\z{1}{0}
\kftext[offset fst=0.6, offset snd=-pi/4]{\zunit}{0}{$1$}
}
}

\kfframes{14}{ % : 46 - :00

  \newcommand\ta{0} %% BOTH WITH MODULUS ...
  \newcommand\tb{3} %% THEIR ARGUMETNS ARE ...
  \newcommand\tdelay{1.5}
  \newcommand\td{13}
  \zgrid[unit=3]{
    \zout[in=0, out=\td+\dt]{1}{0}
    \kftextout[out=\td+\dt, duration=\dt, offset fst=0.6, offset snd=-pi/4]{\zunit}{0}{$1$};

    \kftween[in=\ta, duration=\dt]{0}{2*pi}
    \pgfmathsetmacro\th\pgfmathresult
    \kftween[in=\td, duration=\dt]{2*pi}{0}
    \pgfmathsetmacro\thb\pgfmathresult
    \pgfmathsetmacro\th{\t > \td ? \pgfmathresult : \th}

    
    \pgfkeys{/pgf/fpu=false}
    \draw[dashed] (\zunit, 0) arc (0:\th r:\zunit);
    \pgfkeys{/pgf/fpu=true}

    \zin[in=\ta+\dt, out=\td, arg visible=false]{1}{2*pi/3}
    \zin[in=\ta+2*\dt, out=\td, arg visible=false]{1}{-2*pi/3}

    \zargin[in=\tb, out=(\tb+\tdelay)]{2*pi/3}
    \zargout[in=(\tb+\tdelay), out=(\tb+\tdelay+\dt)]{2*pi/3}
    \kftextin[duration=\dt, in=(\tb+\dt), out=(\tb+\tdelay)]{1.8}{pi/4}{$\frac{2}{3}\pi$}
    \kftextout[duration=\dt, in=(\tb+\tdelay), out=(\tb+\tdelay+\dt)]{1.8}{pi/4}{$\frac{2}{3}\pi$}

    \kftextin[duration=\dt, in=\tb+\tdelay, out=\td, offset fst=1, offset snd=0.9*pi]{\zunit}{2*pi/3}{$e^{i\frac{2}{3}\pi}$}

    
    \pgfmathsetmacro\tb{\tb+3}
    \zargin[in=\tb, out=(\tb+\tdelay)]{-2*pi/3}
    \zargout[in=(\tb+\tdelay), out=(\tb+\tdelay+\dt)]{-2*pi/3}
    \kftextin[duration=\dt, in=(\tb+\dt), out=(\tb+\tdelay)]{1.8}{-pi/4}{$-\frac{2}{3}\pi$}
    \kftextout[duration=\dt, in=(\tb+\tdelay), out=(\tb+\tdelay+\dt)]{1.8}{-pi/4}{$-\frac{2}{3}\pi$}

    \kftextin[duration=\dt, in=\tb+\tdelay, out=\td, offset fst=1.3, offset snd=1.1*pi]{\zunit}{-2*pi/3}{$e^{-i\frac{2}{3}\pi}$}

    \kftextout[duration=\dt, in=\td, out=\td+\dt, offset fst=1, offset snd=0.9*pi]{\zunit}{2*pi/3}{$e^{i\frac{2}{3}\pi}$}
    \kftextout[duration=\dt, in=\td, out=\td+\dt, offset fst=1.3, offset snd=1.1*pi]{\zunit}{-2*pi/3}{$e^{-i\frac{2}{3}\pi}$}

    \zout[in=\td, out=\td+\dt, arg visible=false]{1}{2*pi/3}
    \zout[in=\td, out=\td+\dt, arg visible=false]{1}{-2*pi/3}

    
  }
}
            %% : 46 - : 00

\kfframes{15}{
\newcommand\ta{1} % REMEMBER THAT ...

\newcommand\tb{4} % first z out
\newcommand\tc{5}

\newcommand\td{8} % second arg in
\newcommand\targdelay{1}
\newcommand\te{11} % r in
\newcommand\tf{13}
\newcommand\tg{14.5}


\newcommand\zr{1}
\newcommand\za{1}
%% \newcommand\tc{6}
%% \newcommand\td{9}

\zgrid[unit=3]{
\pgfmathsetmacro\zx{\zunit*\zr*cos(\za r)}
\pgfmathsetmacro\zy{\zunit*\zr*sin(\za r)}


\zargin[in=\ta+\dt, out=(\tb)]{\za}
\zin[in=\ta, out=\tb, arg visible=false]{\zr}{\za}

\kfclip{\tb}{\tend}{
\fill[bg-b] (\zx, \zy) circle (0.12);
}

\zout[in=\tb, out=\tc, arg visible=false]{\zr}{\za}
\zargout[in=\tb, out=\tc)]{\za}

\kftextin [in=\ta+\dt, duration=\dt, out=\tb, offset fst=0.3, offset snd=(\zr+pi/2)]{0.7*\zunit}{\za}{$r$}
\kftextin [in=\ta+2*\dt, duration=\dt, out=\tb]{1.5}{0.4}{$\theta$}

\kftextout[in=\tb, duration=\dt, out=\tb+\dt, offset fst=0.3, offset snd=(\zr+pi/2)]{0.7*\zunit}{\za}{$r$}
\kftextout[in=\tb, duration=\dt, out=\tb+\dt]{1.5}{0.4}{$\theta$}

\kftextin[in=\tb, duration=\dt, out=\tf, offset fst=1.7, offset snd=0.2]{\zr*\zunit}{\za}{$e^{i\theta\phantom{(2\pi+\theta)}}$}


\kftexttween[in=\tf, duration=\dt, out=\tg, offset fst=1.7, offset snd=0.2]
{\zr*\zunit}{\za}{$e^{i\theta\phantom{(2\pi+\theta)}}$}
{\zr*\zunit}{\za}{$e^{i(2\pi+\theta)\phantom{\theta}}$}

\kftextout[in=\tg, duration=\dt, out=\tg+\dt, offset fst=1.7, offset snd=0.2]{\zr*\zunit}{\za}{$e^{i(2\pi+\theta)\phantom{\theta}}$}

% second z
\zargin[in=\td, out=\tf]{2*pi}
\zargin[in=\td+\targdelay, out=\tf, arg radius=1.3]{\za}

%\kftextin [in=\td, duration=\dt, out=\te, offset fst=0.3, offset snd=(\zr+pi/2)]{0.5*\zunit}{\za}{$r$}
\kftextin [in=\td+\dt, duration=\dt, out=\tf]{1.6}{2.9}{$2\pi$}
\kftextin [in=\td+\targdelay+\dt, duration=\dt, out=\tf]{1.8}{0.4}{$\theta$}

\zargout[in=\tf, out=\tf+\dt)]{2*pi}
\zargout[in=\tf, out=\tf+\dt), arg radius=1.3]{\za}


\kftextout[in=\tf, duration=\dt, out=\tf+\dt]{1.6}{2.9}{$2\pi$}
\kftextout[in=\tf, duration=\dt, out=\tf+\dt]{1.8}{0.4}{$\theta$}


\zin[in=\te, out=\tg, arg visible=false]{\zr}{\za}
\zout[in=\tg, out=\tg+\dt, arg visible=false]{\zr}{\za}

\kftextin [in=\te, duration=\dt, out=\tf, offset fst=0.3, offset snd=(\zr+pi/2)]{0.7*\zunit}{\za}{$r$}
\kftextout[in=\tf, duration=\dt, out=\tf+\dt, offset fst=0.3, offset snd=(\zr+pi/2)]{0.7*\zunit}{\za}{$r$}

}
}
               %% : 00 - : 19

\kfframes{8}{                                     %% : 19 - : 27
  \zgrid[unit=3]{
    \zin[arg visible=false]{1}{0}
    \zin[arg visible=false, in=\dt]{1}{2*pi/3}
    \zin[arg visible=false, in=2*\dt]{1}{-2*pi/3}


    \kftween[in=\dt, duration=\dt]{0}{2*pi}
    \pgfmathsetmacro\th\pgfmathresult
    \pgfkeys{/pgf/fpu=false}
    \draw (\zunit, 0) arc (0:\th r:\zunit);
    \pgfkeys{/pgf/fpu=true}

    \kftextin[in=2, duration=\dt, out=\tend]{1}{pi/2}{\color{white}{\colorbox{black}{\phantom{Third roots of unity.}}}}
    \kftextin[in=2, duration=\dt, out=\tend]{1}{pi/2}{\color{white}{Third roots of unity.}}
  }
}
