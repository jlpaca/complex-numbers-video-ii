\kfframes{7}{

  \newcommand\taa{0} % scale
  \newcommand\ta{1} % scale& show t counter
  \newcommand\tb{2} % particle moves
  \newcommand\tc{5} % clip
  \newcommand\td{6} % fadeout
  
  % phantom spaghetti
  \kftext[polar=false]{\halfwidth-1.6}{6}{$\begin{aligned}\phantom{\frac{df}{dt}} &\phantom{=} \;\:c \phantom{f}\\\phantom{f(0) }&\phantom{= 1}\end{aligned}$}
  \kftext[polar=false]{\halfwidth}{6.5}{$\phantom{c} > 0$}

  \pgfmathsetmacro\tt{max(0, \t-\tb-\dt)}
  \kftextin[in=\ta, duration=\dt, out=\td, polar=false]{\halfwidth}{5.5}{$t = \pgfmathprintnumber[fixed,precision=3, fixed zerofill=true]{\tt}$}
  \kftextout[in=\td, out=\td+\dt, polar=false]{\halfwidth}{5.5}{$t = \pgfmathprintnumber[fixed,precision=3, fixed zerofill=true]{\tt}$}


  \kftween[in=\taa, duration=\dt]{3}{6}
  \pgfmathsetmacro\ssx\pgfmathresult
  \begin{scope}[shift={({\halfwidth-\ssx}, {\halfheight-2})}]

    % zoom out yo
    \kftween[in=\taa, duration=\dt]{3}{1}
    \pgfmathsetmacro\zunit{\pgfmathresult}

    % axis & leftovers
    \draw[bg-b] (-\halfwidth+\ssx, 0) -- ({\halfwidth+\ssx}, 0);

    \pgfmathsetmacro\xfo{0.16}

    % fade out the line too
    \fill[bg-b] (\zunit, 0) circle (0.12);

    \kfclip{\tc}{\tend+1}{
      \kftween[in=\td, duration=\dt]{100}{0}
      \pgfmathsetmacro\cxc\pgfmathresult
      \draw[fg-a!\cxc!bg-b] (0, 0) -- ({\halfwidth+\ssx}, 0);
    }

    \fill (0, 0) circle (0.12);

    \kftext[offset fst=0.6, offset snd=3*pi/4]{0}{0}{$0$}
    \kftextin[in=\ta, duration=\dt, out=\tend+1, offset fst=0.6, offset snd=3*pi/4]{\zunit}{0}{$1$}
    
    % position & velocity vectors & marker
    \kfclip{0}{\tc}{
      \pgfmathsetmacro\xx{\zunit*e^(1.8*\tt)}

      \fill (\xx, 0) circle (0.12);
      
      \draw[->, >=stealth] (\xfo, 0) -- (\xx-\xfo, 0);    
      \kftext[offset fst=0.7, offset snd=-pi/2]{0.5*\xx}{0}{$f$}

      \pgfmathsetmacro\dxl{\zunit/4}
      \pgfmathsetmacro\xas{(2*\dxl+\zunit)/\zunit}

      \kftext[polar=false]{\xx+\xx*\xas/2}{0.8}{$\frac{df}{dt}$}
      \draw[->, >=stealth] (\xx, 0) -- ({\xx*(1+\xas)-\xfo}, 0);
    }


  \end{scope}
}
